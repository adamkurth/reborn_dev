\documentclass[12pt]{article}
\usepackage{amsmath}
\usepackage{amssymb}
\usepackage{graphicx}
\usepackage{geometry}
\usepackage{enumitem}
\geometry{
a4paper,
left=20mm,
right=20mm,
top=20mm,
bottom=20mm,
}
\usepackage[makeroom]{cancel}
\usepackage{color}
\usepackage{hyperref}
\usepackage[utf8]{inputenc}
\usepackage[T1]{fontenc}


\setlength{\parindent}{0ex}
\setlength{\parskip}{10pt}

\renewcommand{\vec}[1]{\boldsymbol{#1}}
\newcommand{\hvec}[1]{\hat{\vec{#1}}}
\newcommand{\avg}[1]{\left\langle #1 \right\rangle}
\newcommand{\abs}[1]{\left| #1 \right|}
% Griffiths "script r"
\def\rcurs{{\mbox{$\resizebox{.09in}{.08in}{\includegraphics[trim= 1em 0 14em 0,clip]{fonts/ScriptR}}$}}}
\def\brcurs{{\mbox{$\resizebox{.09in}{.08in}{\includegraphics[trim= 1em 0 14em 0,clip]{fonts/BoldR}}$}}}
\def\hrcurs{{\mbox{$\hat \brcurs$}}}


\title{Scattering intensity under the Born approximation.}
\author{Richard Kirian}
\date{\today}

\begin{document}
\maketitle

\section{Overview}

There are multiple ways to develop the theory of x-ray diffraction.  In order to make this guide 
accessible to motivated undergraduates, we will build upon the most popular
undergraduate electrodynamics textbook by David Griffiths\cite{Griffiths2018}.  The main result that is 
derived below is the following equation for the measured diffraction intensity:
\begin{align}\label{eqn:key}
    I(\vec{q}) = J_0 r_e^2 \Delta \Omega P(\vec{q})  \abs{\int  \rho(\vec{r}) e^{-i \vec{q}\cdot\vec{r}} d^3 r }^2 \;.
\end{align}
The above result assumes (1) perfect plane-wave illumination with wavelength $\lambda$,
and (2) the first-order Born approximation, which loosely means that the scattering amplitudes are very weak 
compared to the incident amplitudes. The quantities in equation \ref{eqn:key} are as follows:
\begin{itemize}
\item $I(\vec{q})$ is the number of photons detected in a detector pixel
\item $\vec{q}=\vec{k}-\vec{k}_0$ is the wavevector transfer
\item $\vec{k}_0$ is the incident wavevector with magnitude $|\vec{k}_0| = |\vec{k}| = \frac{2\pi}{\lambda}$
\item $\vec{k}$ is the outgoing wavevector that points from the target to the detector pixel
\item $I_0$ is the incident intensity (energy per area per time)
\item $\Delta \Omega$ is the solid angle of the detector pixel (assumed very small)
\item $P(\vec{q})$ is the polarization factor defined below
\item $r_e = 2.818 \times 10^{-15}$~m is the Classical electron radius
\item $\rho(\vec{r})$ is the scattering density, which is approximately equal to the electron density.
\end{itemize}

As we can see from equation \ref{eqn:key}, under certain approximations, x-ray diffraction intensities 
allow us to measure the magnitude of the Fourier transform of the scattering density $\rho(\vec{r})$ of the target:
\begin{align}
    \abs{F(\vec{q})} \propto \abs{\int  \rho(\vec{r}) e^{-i \vec{q}\cdot\vec{r}} d^3 r } \;.
\end{align}
Ideally, we would like to measure the complex \emph{amplitudes} $F(\vec{q})$, in which case
we could directly recover the scattering density via inverse Fourier transform:
\begin{align}
\rho(\vec{r}) = \frac{1}{2\pi} \int  F(\vec{q}) e^{i \vec{q}\cdot\vec{r}} d^3 q \;.
\end{align}
The task of recovering the desired molecular structure $\rho(\vec{r})\approx \rho(\vec{r})$ given the 
information-deficient
measurement of $\abs{F(\vec{q})}$ is called the ``phase problem'' in diffraction microscopy and
crystallography.


\section{Electromagnetic fields from a moving point charge}

Here we briefly outline how the electromagnetic 
fields from a moving point charge may be derived starting with Maxwell's equations
in vacuum:
\begin{align}
\nabla \cdot \vec{E} &= \rho/\epsilon_0  &\ \nabla\times\vec{E} &= -\frac{\partial\vec{B}}{\partial t} \\
\nabla \cdot \vec{B} &= 0  & \nabla\times\vec{B} &= \mu_0 \vec{J}+\mu_0 \epsilon_0 \frac{\partial\vec{E}}{\partial t}\;.
\end{align}
As developed in Griffiths chapter 10, Maxwell's equations may be written in terms of the scalar
and vector potentials that are defined as
\begin{align}\label{eqn:pots}
\vec{B} &= \nabla \times \vec{A} & \vec{E} &= -\nabla V -\frac{\partial \vec{A}}{\partial t} \;.
\end{align}
When working in the Lorenz gauge, namely
\begin{align}
\nabla \cdot \vec{A} =  -\mu_0 \epsilon_0 \frac{\partial V}{\partial t} \;,
\end{align}
Maxwell's equations are reduced to a set of four inhomogeneous wave equations:
\begin{align}
\nabla^2 V - \mu_0\epsilon_0 \frac{\partial^2 V}{\partial t^2} &= -\rho / \epsilon_0 & \nabla^2 \vec{A} - \mu_0\epsilon_0 \frac{\partial^2 \vec{A}}{\partial t^2} &= -\mu_0 \vec{J} \;.\label{eqn:ihA}
\end{align}
The inhomogeneous wave equations may be solved using Green's method, which yields the
retarded potentials:
\begin{align}
V(\vec{r}, t) &= \frac{1}{4\pi\epsilon_0}\int \frac{\rho(\vec{r}', t_r)}{\rcurs}d^3r' & \vec{A}(\vec{r}, t) &= \frac{\mu_0}{4\pi}\int \frac{\vec{J}(\vec{r}', t_r)}{\rcurs}d^3r'
\end{align}
where $\brcurs = \vec{r} - \vec{r}'(t_r)$ and the retarded time is $t_r = t -\rcurs/c$.  For a \emph{point charge} 
moving on the \emph{trajectory} $\vec{r}'(t)$, the retarded potentials become the Li\'enard-Wiechert potentials:
\begin{align}
V(\vec{r}, t) &= \frac{1}{4\pi\epsilon_0}    \frac{q}{\left| \rcurs - \brcurs \cdot \vec{\beta} \right|} &
\vec{A}(\vec{r}, t) &= \frac{\mu_0}{4\pi}    \frac{q\vec{v}}{\left| \rcurs - \brcurs \cdot \vec{\beta} \right|} \;,
\end{align}
where $\vec{\beta} = \vec{v}/c = \dot{\vec{r}}'(t_r)/c$. From the Li\'enard-Wiechert potentials, we use equation \ref{eqn:pots} work out the fields for a 
point charge:
\begin{align}\label{eqn:Erad}
\vec{E}(\vec{r}, t)= \frac{q}{4\pi\epsilon_0} \frac{\rcurs}{(\brcurs \cdot \vec{u})^3}
[(c^2 - v^2)\vec{u} + \brcurs \times (\vec{u} \times \vec{a})] \; , \quad\quad \vec{B}(\vec{r}, t)
= \frac{1}{c}\hrcurs \times \vec{E}(\vec{r}, t)
\end{align}
where $\vec{u} \equiv c (\hrcurs - \vec{\beta})$ and $\vec{a}=\ddot{\vec{r}}'(t_r)$.  Importantly, these 
fields are correct for relativistic particles, and as such they may be used to derive both the x-ray scattering
amplitudes as well as the incident source fields created by x-ray free-electron lasers (XFELs) and synchrotrons.


\section{Dipole radiation}

Under the dipole approximations we consider locations very far from a localized source.  The
radiation field in equation \ref{eqn:Erad} dominates because it falls off in proportion to $1/\rcurs$ whereas the generalized
Coulomb field falls off in proportion to $1/\rcurs^2$.  We therefore focus on the
radiation field
\begin{align}
\vec{E}_\text{rad} = \frac{q}{4\pi\epsilon_0} \frac{\rcurs}{(\brcurs\cdot \vec{u})^3}
[\brcurs \times (\vec{u} \times \vec{a})]
\end{align}
for the remainder of this note.

Consider a point charge that oscillates about the origin with position $\vec{r}' = r_0 \cos(-i\omega t) \hvec{r}'$.
We make the standard dipole radiation assumptions: the dipole is small compared to the distance
to our observation point, which means that $r \gg r'$ and hence $\brcurs \approx \vec{r}$.
We further assume that the charged particle is non-relativistic, which means that\footnote{Equivalently, 
$r_0\omega/c = 2\pi r_0/\lambda \ll 1 $; the wavelength is much
larger than the dipole, and the time it takes light to traverse the extent of the dipole is
small compared to the period of oscillation.}
$v \ll c$, and we may use $\vec{u} \approx c \hrcurs$.  Plugging these in, we get
\begin{align}\label{eqn:dip1}
\vec{E}_\text{dip}(\vec{r},t) &= \frac{q}{4\pi\epsilon_0} \frac{r}{(\vec{r}\cdot c \hvec{r})^3}
\vec{r} \times (c \hvec{r} \times \vec{a}) \\
&= \frac{\mu_0 }{4\pi} \frac{1}{r} [\hvec{r} \times ( \hvec{r} \times  \ddot{\vec{p}}(t_r))
\end{align}
where $\vec{p}(t) = q \vec{r}'(t)$ is the dipole moment of the point charge.
The magnetic field is
\begin{align}
\vec{B}_\text{dip} =  \frac{1}{c} \hvec{r} \times \vec{E}_\text{dip} 
\end{align}
and the pointing vector is
\begin{align}
\vec{S} =  \frac{1}{\mu_0} \vec{E}_\text{dip}\times\vec{B}_\text{dip} =  \frac{1}{\mu_0c} E_\text{dip}^2\hvec{r} 
= \frac{\mu_0}{16 \pi^2 cr^2} |\hvec{r} \times    \ddot{\vec{p}}|^2\hvec{r} \;.\\
\end{align}
The differential power radiated through a differential area $d\vec{a} = r^2 \sin\theta d\theta d\phi \hvec{r} = r^2 d\Omega \hvec{r}$ is
\begin{align}
dP = \vec{S}\cdot d\vec{a} = \frac{\mu_0 }{16\pi^2 c} |\hvec{r} \times    \ddot{\vec{p}}|^2 d\Omega \; .
\end{align}
The total power radiated away is
\begin{align}
P = \int \frac{dP}{d\Omega} d\Omega =  \frac{\mu_0 }{16\pi^2 c}  \int |\hvec{r} \times    \ddot{\vec{p}}|^2 d\Omega  \; .
\end{align}
The above integral does not depend on the orientation of the coordinate system, and is most easily performed if we choose $|\hvec{r} \times \hvec{p}|^2 = \sin^2\theta$ which yields a factor of $8\pi/3$.  The final result is the Larmor formula:
\begin{align}
P =  \frac{\mu_0 \ddot{p}^2}{6\pi c}  \; .
\end{align}
For sinusoidal oscillations, a time average of the above introduces a factor of $1/2$:
\begin{align}
\avg{dP} &= \frac{\mu_0 \omega^4 p^2}{32\pi^2 c} |\hvec{r}\times\hvec{p}|^2 d\Omega \\
\langle P \rangle &=  \frac{\mu_0 \omega^4 p^2}{12\pi c} \; . \label{eqn:avgpwer}
\end{align}


\section{Thompson scatter from a free electron}

An example of a very simple classical scattering process is a free electron exposed to an electromagnetic wave, which results in acceleration and hence re-radiation or \emph{scattering}.    
The force on an electron (mass $m_e$ and charge $e$) due to an incident light beam with oscillating electric field $\vec{E}_0(t)$ 
results in the acceleration
\begin{align}
\vec{a}(t) =  -\frac{e}{m_e} \vec{E}_0(t) \;.
\end{align}
If the electron was bound to an atom we could develop a crude model by adding a restoring force and damping force as in Griffiths chapter 9.  In our simple model for a \emph{free} electron, the second derivative of the dipole moment is 
\begin{align}
\ddot{\vec{p}} =  -e \vec{a} = \frac{e^2}{m_e} \vec{E}_0(t) \;.
\end{align}  
Plugging this into the dipole radiation formula \ref{eqn:dip1} gives
\begin{align}
\vec{E}_\text{dip} =  -\frac{r_e}{r} [\hvec{r} \times ( \hvec{r} \times  \vec{E}_0(t_r))] 
\end{align}
where we simplified the expression by defining the classical electron radius
\begin{align}
r_e = \frac{1}{4\pi \epsilon_0} \frac{e^2}{m_e c^2} \approx 2.8\times10^{-15} \; \text{m} \;.
\end{align}
We therefore have 
\begin{align}
\avg{dP}  = \frac{r_e^2 E_0^2}{ 2 \mu_0 c}  |\hvec{r}\times\hvec{E}_0|^2 d\Omega  = r_e^2  |\hvec{r}\times\hvec{E}_0|^2 I_0 d\Omega   \; .
\end{align}
where $I_0 = |\hvec{E}_0|^2/2\mu_0c$ is the incident radiation intensity.  Integrating over all angles we get
\begin{align}
P = \frac{8}{3}\pi r_e^2  I_0  = \sigma_T  I_0 
\end{align}
where $\sigma_T$ is the Thompson scattering cross section.

\section{Weak Diffraction (Born approximation)}

Consider the the scattered field from a free electron located at position $\vec{r}'$ under the influence of a plane wave
\begin{align}
 \vec{E}_0(\vec{r},t) = \vec{E}_0 \exp(i\vec{k}_0 \cdot \vec{r} - i \omega t)
\end{align}
The scattered field is
\begin{align}
\vec{E}_\text{dip}(\vec{r},t) =  \frac{r_e}{r} [\hvec{r} \times ( \hvec{r} \times  \vec{E}_0 )] \exp(i\vec{k}_0 \cdot \vec{r}' - i \omega t_r)
\end{align}
where $t_r = t - \rcurs/c$.  Since we observe in the ``far-field'', where $r \gg r'$, we may use the following approximation:
\begin{align}
 \rcurs = \sqrt{\vec{r} - \vec{r}'} = \sqrt{r^2 + r'^2 -2\vec{r}\cdot \vec{r}'} \approx r  - \hvec{r}\cdot \vec{r}' \;.
\end{align}
Plugging in this approximation we have
\begin{align}
\vec{E}_\text{dip}(\vec{r},t) =  \frac{r_e}{r} [\hvec{r} \times ( \hvec{r} \times  \vec{E}_0 )]\exp\left(i\vec{k}_0 \cdot \vec{r}' - i\omega t + i \frac{\omega}{c}( r  - \hvec{r}\cdot \vec{r}')\right) \;.
\end{align}
Defining $\vec{k} = \omega/c\; \hvec{r}=2\pi/\lambda \hvec{r}$ (the outgoing wavevector, directed at the detector) and $\vec{q} = \vec{k}-\vec{k}_0$, we get
\begin{align}
\vec{E}_\text{dip}(\vec{r},t) =  r_e\frac{e^{ikr}}{r} e^{-i\omega t}[\hvec{r} \times ( \hvec{r} \times  \vec{E}_0 )]e^{-i\vec{q}\cdot\vec{r}'} \;.
\end{align}
Finally, in order to form a diffraction pattern, we sum over many electrons at positions $\vec{r}_i'$:
\begin{align}
\vec{E}_\text{diff}(\vec{r},t) =  r_e\frac{e^{ikr}}{r} e^{-i\omega t}[\hvec{r} \times ( \hvec{r} \times  \vec{E}_0 )] \sum_i e^{-i\vec{q}\cdot\vec{r}_i'} \;.
\end{align}
In the continuum limit, with electron density $\rho(\vec{r})$, we have
\begin{align}
\vec{E}_\text{diff}(\vec{r},t) =  r_e\frac{e^{ikr}}{r} e^{-i\omega t}[\hvec{r} \times ( \hvec{r} \times  \vec{E}_0 )] \int \rho(\vec{r}') e^{-i\vec{q}\cdot\vec{r}'} d^3r' \;.
\end{align}
We typically define the ``form factor'' as
\begin{align}
 F(\vec{q}) = \int \rho(\vec{r}') e^{-i\vec{q}\cdot\vec{r}'} d^3r' \;.
\end{align}
This form factor is just the Fourier transform of the electron density, with the conjugate variable $\vec{q}$ being defined by the incoming and outgoing wavevectors.  The time-averaged Poynting vector is\footnote{See Appendix \ref{sec:poynt} for details.}
\begin{align}
\avg{\vec{S}(\vec{r})} = \frac{1}{2\mu_0} \Re\{\vec{E}_\text{dip}\times\vec{B}_\text{dip}^*\} = r_e^2 \frac{1}{r^2} I_0 | \hvec{r} \times  \hvec{E}_0 |^2 \left| F(\vec{q}) \right|^2 \hvec{r} \;.
\end{align}
Where we have again used the incident intensity $I_0 = |\vec{E}_0|^2/2\mu_0$.  Assuming a detector pixel with area $d\vec{a} = r^2 d\Omega \hvec{r}$, the power into the detector pixel is
\begin{align}
I(\vec{q}) = \avg{\vec{S}(\vec{r})} \cdot d\vec{a} = r_e^2 I_0 | \hvec{r} \times  \hvec{E}_0 |^2 \left| F(\vec{q}) \right|^2 d\Omega\;.
\end{align}
The ``polarization factor'' is usually defined as $P=| \hvec{r} \times  \hvec{E}_0 |^2=\sin^2\theta$, and in this case the polarization factor corresponds to linear polarization along the direction $\hvec{E}_0$.  We may allow for the possibility of
elliptical polarization if we write the field as a superposition of two orthogonal polarizations:
\begin{align}
 \vec{E}_0 = \vec{E}_1 + e^{i\phi}\vec{E}_2 
\end{align}
where $\hvec{E}_2 = \hvec{k}_0 \times \hvec{E}_1$.  The time-averaged Poynting vector becomes\footnote{See Appendix \ref{sec:poynt} for details.}
\begin{align}
\avg{\vec{S}(\vec{r})} &= r_e^2 \frac{1}{r^2} (I_1 | \hvec{r} \times  \hvec{E}_1 |^2 + I_2 | \hvec{r} \times  \hvec{E}_2 |^2) \left| F(\vec{q}) \right|^2 \hvec{r} \;. 
\end{align}
If we write the incident intensity as 
\begin{align}
 I_0 = I_1 + I_2 
\end{align}
and parameterize the intensity components according to the weight $\alpha$ such that
\begin{align}
 I_1 &=\alpha I_0\\
 I_2 &= (1-\alpha)I_0
\end{align}
we may write the diffraction intensity as
\begin{align}
I(\vec{q}) = r_e^2 I_0 (\alpha | \hvec{r} \times  \hvec{E}_1 |^2 + (1-\alpha) | \hvec{r} \times  \hvec{E}_2 |^2) \left| F(\vec{q}) \right|^2 d\Omega\;.
\end{align}
We have now arrived at equation \ref{eqn:key}:
\begin{align}
    I(\vec{q}) = J_0 r_e^2 \Delta \Omega P(\vec{q})  \abs{\int  \rho(\vec{r}) e^{-i \vec{q}\cdot\vec{r}} d^3 r }^2
\end{align}
where the polarization factor is defined as
\begin{align}\label{eqn:goodpol}
 P(\vec{q}) = \alpha | \hvec{k} \times  \hvec{E}_1 |^2 + (1-\alpha) | \hvec{k} \times \hvec{E}_2 |^2
\end{align}
with the understanding that 
\begin{align}
 \hvec{r} = \hvec{k} = \frac{\vec{q} - \vec{k}_0}{\abs{\vec{q} - \vec{k}_0}} \;.
\end{align}
A convenient form for calculations is
\begin{align}
 P(\vec{q}) &= 1 - \alpha(\hvec{k} \cdot \hvec{E}_1 )^2 - (1-\alpha)(\hvec{k} \cdot \hvec{E}_2 )^2 \;.
\end{align}

% \section{Rayleigh scatter from a dielectric sphere}
% 
% Outside of the weak-scattering approximation, diffraction is a complicated process.  There are few exact analytic solutions in such cases.  However, if the object is much smaller than the wavelength of the light, then our dipole approximations can be used, provided that we can determine the relationship between the incident field $\vec{E}_0$ and the dipole that it induces:
% \begin{align}
% \vec{p} = \alpha \vec{E} \;,
% \end{align}
% where $\alpha$ is sometimes referred to as the polarizability of the object.  For asymmetric objects, the induced dipole vector will point in a direction that differs from the polarization of the incident field, in which case $\alpha$ would take the form of a $3\times3$ tensor.  
% 
% In Griffiths chapter 4, you considered a dielectric sphere with relative permittivity $\epsilon_r$ placed in a uniform field, and found that the internal field $\vec{E}_0$ within the sphere is 
% \begin{align}
% \vec{E}_\text{int} = \frac{3}{\epsilon_r + 2} \vec{E}_0 \;.
% \end{align}
% Assuming that the sphere is much smaller than the wavelength of the radiation, we may use this result to determine the Rayleigh scattering cross section for a dielectric sphere.  The polarization (dipole moment per volume) is $\vec{P}=\epsilon_0\chi_e \vec{E}_\text{int} = \epsilon_0(\epsilon_r - 1) \vec{E}_\text{int}$, and hence the total dipole moment of a sphere of radius $R$ is
% \begin{align}
% \vec{p} = \frac{4}{3} \pi R^3 \vec{P} = 4 \pi R^3 \epsilon_0  \frac{\epsilon_r-1}{\epsilon_r + 2} \vec{E}_0 \;.
% \end{align}
% Next we plug this dipole moment into our differential radiated power (equation \ref{eqn:avgpwer}):
%\begin{align}
%\left\langle \frac{dP}{d\Omega} \right\rangle &= \frac{\mu_0  \epsilon_0^2 \omega^4 }{2 c} |\hvec{r}\times\hvec{p}|^2   R^6  \left(\frac{\epsilon_r-1}{\epsilon_r + 2}\right)^2 E_0^2 \; .
%\end{align}
%We relate the above to the incident intensity (power per area) $I_0 = \frac{1}{2\mu_0 c} E_0^2$ to get
%\begin{align}
%\left\langle \frac{dP}{d\Omega} \right\rangle &= \frac{\omega^4}{2c^4}  |\hvec{r}\times\hvec{p}|^2   R^6  \left(\frac{\epsilon_r-1}{\epsilon_r + 2}\right)^2 I_0 \; .
%\end{align}
%We can integrate over all solid angles to get
% \begin{align}
% P  &= \frac{\omega^4}{2c^4}  \frac{8}{3}\pi   R^6  \left(\frac{\epsilon_r-1}{\epsilon_r + 2}\right)^2 I_0 \; .
% \end{align}
% This is the Rayleigh scattering formula, which may be written as
% \begin{align}
% P &= \sigma_R I_0 \;.
% \end{align}
% The total scattering cross section (area units) is often written in terms of the sphere diameter $D$, wavelength $\lambda$, and refractive index $n$:
% \begin{align}
% \sigma_R = \frac{2}{3} \pi^5   \frac{D^6}{\lambda^4}  \left(\frac{n^2-1}{n^2 + 2}\right)^2 \;.
% \end{align}
% There are tons of nanoparticles in the air that scatter light in this way; that's why you can see a bright laser beam in a dark room.  There are also various density fluctuations in the refractive index of materials that are nominally ``homogeneous and isotropic'' -- Rayleigh scattering qualitatively explains why you can also see scattering from a bright laser even in ultrapure water or air.   The blue sky is partly due to the wavelength dependence of Rayleigh scatter; we see more blue light than red because blue scatters more strongly.
%The Rayleigh scattering equation is relevant to my own research because the $\sim$60nm viruses and $\sim$5nm protein molecules that I shoot with x-rays radiate green laser light according to that formula (within reasonable approximation).  I use to determine what optical laser power is needed to see a high-speed beam of biomolecules as they fly into the femtosecond x-ray beam.
% 
% \section{Attenuation}
% 
% Red sunsets can be understood as a result of light attenuation, caused by wavelength-dependent absorption and scattering processes.  Suppose we have a number density $n$ of scatterers with total scattering cross section $\sigma$.  A very thin slab of area $A$ and infinitesimal thickness $dz$ contains $M = n A dz$ particles within the slab.  Suppose a collimated beam of incident intensity $I(z)$ is directed at the slab, and the intensity that exits the other side of the slab is $I(z+dz)$ (assume this intensity is measured far from the slab, where the detector does not detect the scattered light).  The total power radiated away can be equated as
% \begin{align}
% dP = (I(z+dz) - I(z)) A =  -M \sigma A I(z) \;.
% \end{align}
% Re-arranging, we have
% \begin{align}
% \frac{I(z+dz) - I(z)}{I(z)} =  \frac{dI(z)}{I(z)} = -n\sigma dz 
% \end{align}
% Integrating the above, we have 
% \begin{align}
% I(z) =  I(0) e^{- n  \sigma z}  \;.
% \end{align}
% For a dilute suspension of particles the attenuation length is equal to $z_0 = 1/n\sigma$.  For dense suspensions of particles, we need to consider multiple scattering, but this result holds for particles that absorb with cross section $\sigma$.

%nonetheless these results can explain the blue sky and orange/red sunsets among many other scattering phenomena.  If you put a bit of milk into a jar of water and shine a flahslight through it you can emulate the sunset and blue sky.  The little fat globules in milk are quite large (micrometer scale) and lie in the ``Mie'' scattering regime (the two limits of Mie scattering theory are Rayleigh scatter when $\lambda \gg R$ and the ray optics regime when $\lambda \ll R$), but nonetheless the same effect is seen.  In the ray optics regime, we get rainbows, which you can also calculate, using the Fresnel equations developed in this class.
\newpage

\section{Appendix: Solution to the inhomogeneous wave equation}

We wish to solve for $V(\vec{r}, t)$ in the inhomogeneous equation
\begin{align}\label{eqn:ihV2}
\nabla^2 V(\vec{r}, t) - \mu_0\epsilon_0 \frac{\partial^2 V(\vec{r}, t)}{\partial t^2} &= -\rho(\vec{r}, t) / \epsilon_0
\end{align}
assuming that we know $\vec{\rho}(\vec{r}, t)$.  Note that if we solve this equation, the same approach may be applied to the components
of the vector potential $\vec{A}$ in equation \ref{eqn:ihA}.

Briefly, Green's method works as follows.  Assume we have a linear operator $\mathcal{L}_{\vec{r}}$ and we 
wish to solve for $V(\vec{r})$ in the following equation:
\begin{align}
 \mathcal{L}_{\vec{r}} V(\vec{r}) = g(\vec{r}) \;.
\end{align}
% The fact that the operator is linear means that
% \begin{align}
% \mathcal{L}_{\vec{r}} \big( a V(\vec{r}) + b Q(\vec{r}) \big) =
% a \mathcal{L}_{\vec{r}} V(\vec{r}) + b \mathcal{L}_{\vec{r}} Q(\vec{r})
% \end{align}
% for constants $a$ and $b$.  
Now assume further that we can find the ``Green's function''
$G(\vec{r}, \vec{r}')$ that satisfies the equation
\begin{align}
\mathcal{L}_{\vec{r}} G(\vec{r}, \vec{r}') = \delta(\vec{r}-\vec{r}')
\end{align}
where $\delta(\vec{r}-\vec{r}')$ is the Dirac delta function.  Once equipped with $G(\vec{r}, \vec{r}')$ we may easily
show that the solution for $V(\vec{r})$ is
\begin{align}\label{eqn:useg}
V(\vec{r}) = \int_\text{Vol} g(\vec{r}')G(\vec{r}, \vec{r}') d^3 r'\;.
\end{align}
Working the proof backwards from the above, we apply $\mathcal{L}_{\vec{r}}$ to both sides of the equation:
\begin{align}
\mathcal{L}_{\vec{r}} V(\vec{r}) &= \mathcal{L}_{\vec{r}} \int_\text{Vol} g(\vec{r}')G(\vec{r}, \vec{r}')d^3 r' \\
&= \int_\text{Vol} g(\vec{r}')  \mathcal{L}_{\vec{r}} G(\vec{r}, \vec{r}') d^3 r' \\
    &= \int_\text{Vol} g(\vec{r}')  \delta(\vec{r} - \vec{r}') d^3 r' \\
    \mathcal{L}_{\vec{r}} V(\vec{r}) &= g(\vec{r})
\end{align}

In order to solve equation \ref{eqn:ihV2}, we will first remove the time variable by expanding $V(\vec{r})$ and
$\rho(\vec{r})$ into their Fourier transforms:
\begin{align}
V(\vec{r}, t) &= \int_{-\infty}^{\infty} V_{\omega}(\vec{r})e^{- i\omega t} d\omega \\
    \rho(\vec{r}, t) &= \int_{-\infty}^{\infty} \rho_{\omega}(\vec{r})e^{- i\omega t} d\omega \;.
\end{align}
We plug the above into equation \ref{eqn:ihV2} we get a differential equation for each frequency component:
\begin{align}
(\nabla^2 + k^2) V_\omega(\vec{r}) = -\rho_\omega (\vec{r})/\epsilon_0
\end{align}
where $k = \sqrt{\mu_0\epsilon_0}\omega$.  We now seek the Green's function for
\begin{align}
 (\nabla^2 + k^2)G(\vec{r}, \vec{r}') = \delta(\vec{r}-\vec{r}')
\end{align}
The above is easiest to solve if the delta function is at the origin, so we define $\vec{R} = \vec{r}-\vec{r}'$:
\begin{align}
 (\nabla^2 + k^2)G(\vec{r}, \vec{r}') = \delta(\vec{R})
\end{align}
In the above $\vec{R}$ coordinates the RHS is rotationally invariant, and hence we may write 
$G(\vec{r}, \vec{r}') = G(R)$.  Taking now the radial component of the Laplacian:
\begin{align}\label{eqn:symR}
\left(\frac{1}{R}\frac{\partial}{\partial R}R + k^2\right)G(R) = \delta(R) \;.
\end{align}
If we look in the regions $R \ne 0$ where $\delta(R)=0$ we have the equation
\begin{align}
\frac{\partial}{\partial R}RG(R) =- k^2R G(R)  
\end{align}
with the usual solutions
\begin{align}
 G(R) = A \frac{e^{ikR}}{R} + B \frac{e^{-ikR}}{R} \;.
\end{align}
With foresight, we can set $B=0$ because the \emph{incoming} spherical waves produce results
that violate causality.  In order to determine the coefficient $A$ we integrate equation \ref{eqn:symR}
about an infinitesimal volume centered on the origin:
\begin{align}
\int_\text{sphere}\nabla^2 G(R) d^3R  + k^2 \int_\text{sphere} G(R) d^3R = 1 \;.
\end{align}
In the limit $R\rightarrow 0$ we have $G(R) \rightarrow A/R$:
\begin{align}
A \int_\text{sphere}\nabla^2 \frac{1}{R} d^3R  + A k^2 \int_\text{sphere} \frac{1}{R} 4\pi R^2dR = 1 \;.
\end{align}
From Griffiths chapter 1 we have $\nabla^2 (1/R) = -4\pi \delta(R)$, and the second term on the LHS 
vanishes, finally giving us
\begin{align}
 A = -\frac{1}{4\pi} \;.
\end{align}
We now have the Green's function:
\begin{align}
 G(\vec{r}, \vec{r}') = - \frac{e^{ik|\vec{r}-\vec{r}'|}}{4\pi |\vec{r}-\vec{r}'|}
\end{align}

With the Green's function, we finally plug it into equation \ref{eqn:useg}:
\begin{align}
V_\omega(\vec{r}) = \int_\text{Vol} \frac{\rho_\omega(\vec{r}')}{4\pi \epsilon_0} \frac{e^{ik|\vec{r}-\vec{r}'|}}{ |\vec{r}-\vec{r}'|} d^3 r'\;.
\end{align}
In order to recover our time variable, we take a Fourier transform of the above:
\begin{align}
V(\vec{r}, t) &= \int_{-\infty}^\infty \left\{\int_\text{Vol} \frac{\rho_\omega(\vec{r}')}{4\pi \epsilon_0} \frac{e^{ik|\vec{r}-\vec{r}'|}}{ |\vec{r}-\vec{r}'|} d^3 r' \right\}e^{-i\omega t}d\omega \\
&=\int_\text{Vol} \left\{ \int_{-\infty}^\infty\rho_\omega(\vec{r}') e^{-i\omega (t -|\vec{r}-\vec{r}'|/c)} d\omega\right\}\frac{1}{4\pi \epsilon_0} \frac{1}{ |\vec{r}-\vec{r}'|} d^3 r'  \;.
\end{align}
We can finally write the above as the retarded potential
\begin{align}
V(\vec{r}, t) &= \frac{1}{4\pi\epsilon_0}\int \frac{\rho(\vec{r}', t_r)}{\rcurs}d^3r' \;.
\end{align}

\newpage
\section{Appendix: Li\'enard-Wiechert potential}

We have a particle moving along the trajectory $\vec{w}(t)$, and we want the retarded potential at position $\vec{r}$:
\begin{align}
V(\vec{r}, t) = \frac{1}{4\pi\epsilon_0}\int \frac{\rho(\vec{r}', t_r)}{|\vec{r} - \vec{r}'|} d^3r'
\end{align}
where the retarded time is
\begin{align}
t_r(t, \vec{r}') \equiv t - |\vec{r}-\vec{r}'|/c
\end{align}
The density of a point particle is
\begin{align}
\rho(\vec{r}, t) &= q \delta(\vec{r} - \vec{w}(t)) \\
& = q \int  \delta(\vec{r} - \vec{w}(t')) \delta(t' - t_r(t, \vec{r}'))  dt'
\end{align}
Plug the density into the retarded potential:
\begin{align}
V(\vec{r}, t) = \frac{q}{4\pi\epsilon_0}\iint \frac{  \delta(\vec{r}' - \vec{w}(t')) \delta(t' - t_r(t, \vec{r}'))  }{|\vec{r} - \vec{r}'|} d^3r' dt'
\end{align}
Note that the time $t$ that appears in the retarded time is not the same as the dummy variable $t'$.  Now we do the integral over space, which involves swapping $\vec{r}'$ with $\vec{w}(t)$ everywhere.  In particular, note that there is an 
$\vec{r}'$ in the retarded time $t_r(t, \vec{r}')$
\begin{align}\label{eqn:tint}
V(\vec{r}, t) = \frac{q}{4\pi\epsilon_0}\int \frac{ \delta(t' - t_r(t, \vec{w}(t') )\;)  }{|\vec{r} - \vec{w}(t')|} dt'
\end{align}
Next we need to do the integral over time, but we have a delta function with a function as its argument.  The general rule for a delta function is that it is zero except when the argument is equal to zero.  To handle this situation, we use the 
following general rule:
\begin{align}
\delta(g(x)) &= \frac{\delta(x-x')}{|g'(x')|} 
\end{align}
where $x'$ is implicitly defined by the expression
\begin{align}
g(x') & = 0
\end{align}
and the primed $g'(x)$ means to take the derivative
\begin{align}
g'(x) &\equiv \frac{d }{dx}g(x)  \;.
\end{align}
We therefore can write our delta function in time as:
\begin{align}
\delta(t' - t_r(t, \vec{w}(t') )\;) = \frac{\delta(t' - t^*)}{\left |  \frac{d}{dt'} \left(  t' - \left(t - \frac{|\vec{r} - \vec{w}(t')|}{c}\right) \right)_{t' = t^*} \right |}
\end{align}
We work out the denominator:
\begin{align}
  \frac{d}{dt'} \left(  t' - \left(t - \frac{|\vec{r} - \vec{w}(t')|}{c}\right) \right) &= 1 + \frac{1}{c} \frac{d}{dt'} |\vec{r} - \vec{w}(t')| \\
  &= 1 + \frac{1}{c} \frac{d}{dt'} \left( \left( \vec{r} - \vec{w}(t')\right)^2 \right)^{1/2} \\
  &= 1 + \frac{1}{2c} \frac{1}{ |\vec{r} - \vec{w}(t')| } \frac{d}{dt'}\left( \vec{r} - \vec{w}(t')\right)^2\\
  &= 1 + \frac{1}{2c} \frac{1}{ |\vec{r} - \vec{w}(t')| } \frac{d}{dt'}\left( \vec{r}\cdot \vec{r} + \vec{w}(t')\cdot \vec{w}(t') - 2\vec{r} \cdot \vec{w}(t')\right)\\
&= 1 + \frac{1}{2c} \frac{1}{ |\vec{r} - \vec{w}(t')| } \left( 2 \vec{w}(t')\cdot  \dot{\vec{w}}(t') - 2\vec{r} \cdot 
 \dot{\vec{w}}(t')\right) \\
 &= 1 - \frac{1}{c} \frac{ \left( \vec{r} \cdot 
 \dot{\vec{w}}(t') - \vec{w}(t')\cdot  \dot{\vec{w}}(t') \right)  }{ |\vec{r} - \vec{w}(t')| } \\
 &= 1 - \frac{1}{c} \frac{ \left( \vec{r}  - \vec{w}(t')  \right)\cdot  \dot{\vec{w}}(t') }{ |\vec{r} - \vec{w}(t')| }
\end{align}
We now determine what $t^*$ is by the implicit expression
\begin{align}
t' - \left(t - \frac{|\vec{r} - \vec{w}(t')|}{c}\right) = 0
\end{align}
Compare the above to the definition of the retarded time of the particle:
\begin{align}
t_r = t - \frac{|\vec{r} - \vec{w}(t_r)|}{c}
\end{align}
Thus we have $t^* = t_r$, which we substitute in to get
\begin{align}
V(\vec{r}, t) &= \frac{q}{4\pi\epsilon_0}\int \frac{ 1  }{|\vec{r} - \vec{w}(t')| }\frac{\delta(t' - t_r)}{\left| 1 - \frac{1}{c} \frac{ \left( \vec{r}  - \vec{w}(t')  \right)\cdot  \dot{\vec{w}}(t') }{ |\vec{r} - \vec{w}(t') | } \right|} dt' \\
&= \frac{q}{4\pi\epsilon_0} \frac{ 1  }{|\vec{r} - \vec{w}(t_r)| }      \frac{1}{\left| 1 - \frac{1}{c} \frac{ \left( \vec{r}  - \vec{w}(t_r)  \right)\cdot  \dot{\vec{w}}(t_r) }{ |\vec{r} - \vec{w}(t_r) | } \right|} 
\end{align}
We can now clean this up with the definitions $\dot{\vec{w}}(t_r) = \vec{v}$, $\vec{\beta} = \vec{v}/c$ and $\rcurs = \vec{r} - \vec{w}(t_r)$:
\begin{align}
V(\vec{r}, t) = \frac{1}{4\pi\epsilon_0}    \frac{q}{\left| \rcurs - \brcurs \cdot \vec{\beta} \right|} 
\end{align}
This same procedure may be applied to each of the components of $\vec{A}$ to get
\begin{align}
\vec{A}(\vec{r}, t) = \frac{1}{4\pi\epsilon_0}    \frac{q\vec{v}}{\left| \rcurs - \brcurs \cdot \vec{\beta} \right|} \;.
\end{align}

\section{Appendix: Time-averaged Poynting vector}\label{sec:poynt}

Supposing that we have the electric field $\vec{E}(t)=\vec{E}_0 e^{-i\omega t}$ and corresponding magnetic field $\vec{B}(t)=\vec{B}_0 e^{-i\omega t}=\frac{1}{c} \hvec{k}\times \vec{E}_0e^{-i\omega t}$, and we want the time averaged Poynting vector
\begin{align}
 \avg{\vec{S}} = \frac{1}{T}\int_0^{T}\frac{1}{\mu_0} \Re\{ \vec{E} \}\times \Re\{ \vec{B} \} dt \;,
\end{align}
we must be careful to take the real part of the complex fields.  We can write the real part as
\begin{align}
 \Re\{\vec{E}\} = \frac{1}{2} (\vec{E} + \vec{E}^*)
\end{align}
and plug this into the above:
\begin{align}
 \avg{\vec{S}} = \frac{1}{T}\int_0^{T} \frac{1}{\mu_0}\frac{1}{2} (\vec{E} + \vec{E}^*)\times \frac{1}{2} (\vec{B} + \vec{B}^*) dt \;.
\end{align}
Importantly, the cross terms involving $\vec{E}\times\vec{B}$ and $\vec{E}^*\times\vec{B}^*$ will average to zero because
of the time dependent term $e^{-2i\omega t}$ that remains in the integral.  In the other two terms, there is no time
dependence, and therefore
\begin{align}
 \avg{\vec{S}} = \frac{1}{\mu_0} \frac{1}{4} (\vec{E}\times\vec{B}^* +  \vec{E}^*\times\vec{B}) = \frac{1}{2\mu_0}\Re\{ \vec{E}\times\vec{B}^*\}\;.
\end{align}
In vacuum we have $\vec{B}=\frac{1}{c}\hvec{k}\times \vec{E}$ so that
\begin{align}
 \avg{\vec{S}} = \frac{1}{2c\mu_0}\Re\{ \vec{E}\times(\hvec{k}\times\vec{E}^*)\} = \frac{1}{2c\mu_0} \abs{E}^2 \hvec{k} = I\hvec{k} \;.
\end{align}

If the $E$ field has two orthogonal components $\vec{E}=\vec{E}_1 + \vec{E}_2$ such that $\vec{E}_1\cdot\vec{E}_2=0$ and 
$\vec{B}=\frac{1}{c}\hvec{k}\times \vec{E}$ then
\begin{align}
 \avg{\vec{S}} = \frac{1}{2c\mu_0}\Re\{ (\vec{E}_1 + \vec{E}_2)\times(\hvec{k}\times(\vec{E}_1 + \vec{E}_2)^*)\} \;.
\end{align}
Due to the orthogonality of the field components, the cross terms vanish: e.g.
\begin{align}
 \vec{E}_1 \times(\hvec{k}\times \vec{E}_2^*) = 0 \;.
\end{align}
Therefore the average Poynting vector is
\begin{align}
 \avg{\vec{S}} = \frac{1}{2c\mu_0} ( \abs{\vec{E}_1}^2 + \abs{\vec{E}_2}^2)\hvec{k}\;.
\end{align}

\bibliography{\jobname}
\bibliographystyle{plain}

\end{document}

\documentclass{beamer}
\mode<presentation>{
   \usetheme{Madrid}
%   \usecolortheme{crane}
%   \usecolortheme{whale}
   \usecolortheme{orchid}
   \setbeamercovered{transparent}
   \usenavigationsymbolstemplate{} %get rid of navigation crap
   \setbeamertemplate{footline}[page number]{} %only pagenumbers on bottom
   \setbeamertemplate{footline}{} %Nothing on bottom
}

\usepackage[english]{babel}
\usepackage[latin1]{inputenc}
\usepackage{times}
\usepackage[T1]{fontenc}
\usepackage{amsmath}

%bold vectors
\let\xxxhat\hat
\let\xxxvec\vec
\renewcommand{\hat}[1]{{\boldsymbol {\xxxhat {#1}} }}
\renewcommand{\vec}[1]{\boldsymbol {#1}}

\title{
Efficient rotation with FFTs
}
\author[Kevin E. Schmidt]{
Kevin E. Schmidt
}

\institute[Arizona State University]{
Department of Physics\\
Arizona State University\\
Tempe, AZ 85287 USA
}

%\date[]

\pgfdeclareimage[height=0.5cm]{university-logo}{logoclr}
%\logo{\pgfuseimage{university-logo}}

\begin{document}

\begin{frame}
  \titlepage
%  \raisebox{.06\textheight}[0pt][0pt]{Funding:  }
%  \includegraphics[height=.15\textheight]{nsfe}
\end{frame}

%\begin{frame}{Outline}
%  \tableofcontents
%\end{frame}

\begin{frame}
\frametitle{Rotations as shear}
{\bf Reference:}
``Convolution-based interpolation for fast, high-quality
rotation of images,'' M. Unser, P. Th\`evenaz, and L. Yaroslavsky,
IEEE Transactions on Image Processing, {\bf 4}, 1371 (1995).

\begin{itemize}
\item
The rotation matrix can be decomposed as
\begin{equation*}
\left (
\begin{array}{cc}
\cos\theta & -\sin\theta\\
\sin\theta & \cos\theta\\ 
\end{array}
\right )
=
\left (
\begin{array}{cc}
1 & -\tan\frac{\theta}{2}\\
0 & 1\\ 
\end{array}
\right )
\left (
\begin{array}{cc}
1 &  0 \\
\sin\theta & 1\\ 
\end{array}
\right )
\left (
\begin{array}{cc}
1 & -\tan\frac{\theta}{2}\\
0 & 1\\ 
\end{array}
\right )
\end{equation*}
\item
Notice a matrix like
\begin{equation}
\left (
\begin{array}{cc}
1 & \alpha \\
1 & 0\\
\end{array}
\right ) 
\left (
\begin{array}{c}
x \\
y \\
\end{array}
\right ) = 
\left (
\begin{array}{c}
x +  \alpha y\\
y \\
\end{array}
\right )
\end{equation}
corresponds to a translation of the x coordinate proportional to $y$,
i.e. a shear.
\item
The corresponding translation operator ($\hbar = 1$) is $e^{-i p_x \alpha y}$,
so this result shows that the rotation operator can be written as
\begin{equation}
e^{-i L_z \theta} = e^{-i (x p_y- y p_x)\theta}
=
e^{i p_x y \tan\frac{\theta}{2}}
e^{-i p_y x \sin\theta}
e^{i p_x y \tan\frac{\theta}{2}}
\end{equation}
\end{itemize}
\end{frame}

\begin{frame}
\frametitle{Fourier series review}
\begin{itemize}
\item
We assume our data is localized -- i.e. zero outside some interval of size $L$
\item
We can then make the data periodic outside the interval without changing
its value inside the interval.
\item
Since it is now periodic, we can write it as a Fourier series
\begin{equation}
F(x) = \sum_{k = -\infty}^\infty e^{i \frac{2\pi}{L} k x} \bar F_k \,.
\end{equation}
We can calculate the Fourier coefficients using orthogonality
\begin{equation}
\bar F_k = \frac{1}{L} \int_0^L dx ~ e^{-i \frac{2\pi}{L} k x} F(x)
\end{equation}
where the integral can be over any convenient period.

\end{itemize}
\end{frame}

\begin{frame}
\frametitle{Discrete Fourier transform}
\begin{itemize}
\item
Often we have functions which are ``band limited,'' that is, they
have a maximum $|k|$ value that has nonzero value (or they can be
approximated by such a function.

\item
In that case, we have
\begin{equation}
F(x) = \sum_{k=-k_{\rm max}}^{k_{\rm max}} e^{i \frac{2\pi}{L} kx  } \bar F_k\,.
\end{equation}
This form requires an odd number of non zero $F_k$. An even number can
be accomadated by changing one of the limits to either $-k_{\rm max}-1$ or
$k_{\rm max}+1$ and making the corresponding $\bar F_k$ zero.
\end{itemize}
\end{frame}
\begin{frame}
\frametitle{Discrete Fourier transform}
\begin{itemize}
\item
We can now use the identity for $k$ integer
\begin{equation}
\sum_{j=0}^{N-1} e^{-i \frac{2\pi}{N} k j} = N \delta_{k+nN,0}
\end{equation}
\begin{itemize}
\item
for $k=0$, the exponent is zero, and the result is immediate.
\item
for $k + n N = 0$, the exponent is a multiple of $i 2\pi$, and the result
is again immediate.
\item
for other cases, the sum is a geometric series which sums to
$\frac{1-e^{-i 2\pi k}}{1-e^{-i\frac{2\pi}{N} k}}$.
The numerator is zero and the denominator nonzero, so the identity is shown.
\end{itemize}

\item
For a band limited function, the coefficients can be calculated from
\begin{equation}
\bar F_k = \frac{1}{N} \sum_{j=0}^{N-1} e^{-i\frac{2\pi}{N} k j} F(j L/N) \,.
\end{equation}
\item
Notice this is identical to calculating the Fourier integral using
the trapezoidal rule with $N$ points, where $N$ is the number of
$k$ values.
\end{itemize}
\end{frame}

\begin{frame}
\frametitle{Discrete Fourier transform}
\begin{itemize}
\item
Writing $F_j = F(jL/N)$, and $\bar F_k = N \tilde F_k$, we have
\begin{eqnarray*}
F(x) &=& \frac{1}{N} \sum_{k=k_l}^{k_u} \tilde F_k e^{i \frac{2\pi}{L} k x}
\nonumber\\
F_j &=& F(jL/N)  = \frac{1}{N} \sum_{k=k_l}^{k_u}
\tilde F_k e^{i \frac{2\pi}{N} kj}
\nonumber\\
\tilde F_k &=& \sum_{j=0}^N F_j e^{-i \frac{2\pi}{N} kj} \,.
\end{eqnarray*}
The numpy choice is
$k_l\,,k_u = -\frac{N-1}{2}\,,\frac{N-1}{2}$ for $N$ odd, and
$k_l\,,k_u = -\frac{N}{2}\,,\frac{N}{2}-1$ for $N$ even.
\end{itemize}
\end{frame}

\begin{frame}
\frametitle{Relation to standard libraries}
\begin{itemize}
\item
Standard math libraries calculate this discrete Fourier transformation
as the pairs as
\begin{eqnarray*}
\tilde G_k &=& \sum_{j=0}^{N-1} G_j e^{-i \frac{2\pi}{N} kj} \,.
\nonumber\\
G_j &=& \frac{1}{N} \sum_{k=0}^{N-1} \tilde G_k e^{i \frac{2\pi}{N} kj} \,.
\end{eqnarray*}
with both $0 \le k,j < N$. 
\item
Comparing to our band limited result we identify
\begin{equation}
\tilde F_k = \left \{
\begin{array}{cc}
\tilde G_k & 0 \leq k \leq k_u\\
\tilde G_{k+N} & k_l \leq k < 0\\
\end{array}
\right . \,.
\end{equation}
\end{itemize}
\end{frame}

\begin{frame}
\frametitle{Calculating $F(x)$ from $F_j=F(jL/N)$}
\begin{itemize}
\item
First calculate the discrete Fourier transform
\begin{equation}
{\cal F}_k  = \sum_{j=0}^{N-1} F_j e^{-i\frac{2\pi}{N} jk}\,,
\ \ \ \ 0 \le k < N
\end{equation}
\item
The fftshift function of numpy on ${\cal F}_k$ gives an array 
with the most negative band limited $k = k_l$ as element 0 and
the most positive band limited $k$ as element $N-1$. That is
\begin{equation}
F_k = \left [{\rm fftshift}\left ({\cal F}\right)\right ]_{k-k_l}\,,
\ \ \ \ k_l \leq k \leq k_u
\end{equation}
\item
Evaluating at arbitrary $x$, recall $k_l = -\frac{N- N~{\rm mod}~2}{2}$,
\begin{eqnarray*}
F(x) &=& \frac{1}{N}\sum_{k=0}^{N-1} \left [{\rm fftshift}\left({\cal F}\right)
\right ]_k e^{i\frac{2\pi}{L} (k+k_l) x}
\end{eqnarray*}

\end{itemize}
\end{frame}

\begin{frame}
\frametitle{1-d shift on a grid}
If a function $F(x)$ is peaked at $x=0$, $F(x-a)$ will be peaked at
$a$. To evaluate this shifted function on the grid $jL/N$,
$F^s(j) = F(jL/N-a)$, we need to calculate
\begin{eqnarray*}
F^s_j &=&
\frac{1}{N}\sum_{k=0}^{N-1}
\left [{\rm fftshift}\left({\cal F}\right) \right ]_k
e^{i\frac{2\pi}{L} (k+k_l) (\frac{jL}{N}-a)}
\nonumber\\
&=&
e^{i\frac{2\pi}{N} k_l (j-\frac{Na}{L})}
\frac{1}{N} \sum_{k=0}^{N-1}
e^{-i\frac{2\pi}{N} k \frac{Na}{L}}
\left [{\rm fftshift}\left({\cal F}\right) \right ]_k
e^{i \frac{2\pi}{N} kj}
\nonumber\\
&=&
e^{-i\pi (1-\frac{N~{\rm mod}~2}{N}) (j-\frac{Na}{L})}
\frac{1}{N} \sum_{k=0}^{N-1}
e^{-i\frac{2\pi}{N} k \frac{Na}{L}}
\left [{\rm fftshift}\left({\cal F}\right) \right ]_k
e^{i \frac{2\pi}{N} kj}
\end{eqnarray*}
\end{frame}
\begin{frame}
\frametitle{Rotation from shift}
\begin{itemize}
\item
We first calculate $F(x-y\tan\frac{\theta}{2},y)$. That is we 
perform the $x$ translation proportional to $y$
as above on each set of fixed $y$ values
translating the $x$ coordinate.
\item
We then translate the $y$ coordinates similarly.
\item
Repeat the translation of $x$ coordinates.

\end{itemize}
\end{frame}

\end{document}

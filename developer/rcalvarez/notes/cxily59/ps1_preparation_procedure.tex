\documentclass[12pt]{article}
\usepackage{amsmath}
\usepackage{amssymb}
\usepackage{geometry}[1 in]
\usepackage{physics}

\title{PS1 Aqueous Sample Preparation Procedure}
\author{Roberto Alvarez, Richard Kirian, and Petra Fromme}
\date{\today}

\begin{document}
    \maketitle

    \section{Overview}
    This note details how to prepare a PS1 solution from PS1 crystals. Provide 2-3 hours.

    \section{Preparation}
    We will need:
    \begin{itemize}
        \item Nine 1.5 mL eppendorf tube (epi)
        \item 1000 $\mu$L pipette
        \item 100 $\mu$L pipette
        \item 10 $\mu$L pipette
        \item Appropriate pipette tips
        \item Settled PS1 crystals
        \item 500 mL $G_{0}$
        \item 500 mL $G_{100}$
        \item Centrifuge
        \item Microscope (optional)
        \item UV-Vis Spectrophotometer
        \item Calculator (recommended)
        \item Lab notebook (always take notes!)
    \end{itemize}

    \section{Buffers}
    \subsection{500 mL $G_{0}$}
    5 mM MES pH 6.4; 0.02 \% $\beta$DDM
    \begin{enumerate}
        \item 450 mL nanopure H$_{2}$O
        \item 6.25 mL 400 mM MES pH 6.4
        \item 1 mL 10\% $\beta$DDM
        \item Fill to volume
    \end{enumerate}
    \subsection{500 mL $G_{100}$}
    5 mM MES pH 6.4, 100 mM MgSO$_{4}$; 0.02 \% $\beta$DDM
    \begin{enumerate}
        \item 400 mL nanopure H$_{2}$O
        \item 50 mL 1 M MgSO$_{4}$
        \item 6.25 mL 400 mM MES pH 6.4
        \item 1 mL 10\% $\beta$DDM
        \item Fill to volume
    \end{enumerate}

    \newpage

    \section{Drying PS1 Crystals}
    \begin{enumerate}
        \item Weigh one 1.5 mL eppendorf tube.
        \item Remove 90\% of the super-native solution from your crystal solutions.
        \item Move the crystals to the weighed eppendorf tube.
        \item Centrifuge the crystals at 14 krpm for 4 minutes.
        \begin{itemize}
            \item Place the tab of the eppendorf tube facing away from the axis of rotation. This will ensure we always know where the crystal pellet is.
            \item Do not forget to balance the crystals with an equal volume of water.
        \end{itemize}
        \item Remove 90\% of the remaining super-native solution.
        \item Repeat centrifuging and solution removal 2 more times.
        \begin{itemize}
            \item We want the crystal pellet to be ``bone-dry''
        \end{itemize}
        \item Weigh the eppendorf + crystal pellet.
    \end{enumerate}
    \begin{equation}
        w_{\mathrm{pellet}} = w_{\mathrm{epi + pellet}} - w_{\mathrm{epi}}
    \end{equation}

    \section{Dissolving PS1 Crystals}
    \begin{enumerate}
        \item Add an equal amount of $G_{100}$ in $\mu$L to the weight of your pellet in mg.
        \item Wash the pellet with the $G_{100}$.
        \begin{itemize}
            \item Pipette small amount of the $G_{100}$ from the side which does not have the pellet (opposite the eppendorf tube tab).
            \item Bring the pipette directly over the pellet (without touching the pellet) and release the solution.
            \item Wash until the pellet is no longer visible.
        \end{itemize}
        \item Let the solution stand for 30 minutes.
    \end{enumerate}

    \section{Optional: Testing Dissolved Crystals}
    \begin{enumerate}
        \item Add a small droplet (0.5 $\mu$L is fine) of the dissolved crystals to a microscope slide.
        \item Under the microscope you should no longer see crystals.
        \item Add a small droplet of $G_{0}$ to re-grow the crystals.
        \begin{itemize}
            \item Place the two droplet directly next to each other so there are only just touching.
            \item Crystals should begin to form within 1-2 minutes.
        \end{itemize}
    \end{enumerate}

    \section{Determining Chlorophyll Concentration}
    \begin{enumerate}
        \item Prepare 2 epis with 80\% acetone (800 $\mu$L) and 20\% H$_{2}$O (2 $\mu$L)
        \item Prepare 4 epis with 80\% acetone (800 $\mu$L) and 19.5\% H$_{2}$O (19.5 $\mu$L)
        \item Add 0.5 $\mu$L PS1
        \begin{itemize}
            \item \textbf{Option 1 (0.5 $\mu$L Capillary):} Grab capillary with calipers and place in the PS1 solution. Pull capillary out and clean with fiber-free cloth. Break capillary in half, drop in acetone-water solution, and vortex immediately until there is no green solution left in the capillary.
            \item \textbf{Option 2 (Pipette):} Pipette 0.5 $\mu$L then dispense into the acetone-water solution. Before all the solution has left the pipette drop the pipette into the acetone-water epi. Vortex immediately until there is no green solution left in the pipette.
        \end{itemize}
        \item Centrifuge at 14 krpm for 4 minutes to remove denatured proteins.
        \item Calibrate UV-Vis spectrophotometer with the two blank samples (no PS1).
        \item Record absorption at 664 nm and 700 nm for the four prepared samples.
        \item Dilute sample to desired concentration.
    \end{enumerate}

    \subsection*{Notes on Pipetting Acetone}
    \begin{itemize}
        \item Acetone must be pipetted once to equalize the vapor pressure in pipette
        \item The pressure change from pipetting will evaporate some acetone, so the amount you set is never the amount that is lifted by the pipette.
        \item Always measure the mass of the acetone.
        \item 635 mg $\equiv$ 800 $\mu$L of Acetone
    \end{itemize}
\end{document}
